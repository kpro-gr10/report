\chapter{Installation Guide}

This chapter contains the instructions for how to install and run the game on 
Android and iOS.

\section{Android}

This section will describe how to install the game on an Android phone.

\subsection*{Manual Installation}
\begin{enumerate}

	\item{} You need an application that allows you to explore the Android file
	system. If your Android device doesn't already come with an application, 
	we recommend "ES File Explorer" or "ASTRO File Manager".

	\item{} Connect your Android phone to your computer, using the Android Micro
	USB cable.

	\item{} Copy the apk file from your computer to any folder on your Android 
	phone.

	\item{} Use the file explorer application you chose to navigate to the folder 
	where you placed the apk file.

	\item{} Select the apk file and click install.

	\item{} The application is now fully installed. Click done to return to the 
	file explorer, or click open to run the game now.

\end{enumerate}

\subsection*{Android Marketplace}
Please refer to the Android distribution guide for instructions on how to put 
the application on the Android market place, as of 19.11.2013 this can be found 
on the following webpage: \href{http://developer.android.com/distribute/index.html}{developer.android.com/distribute/index.html}

\section{iOS}

The iOS operating system does not allow ordinary users to install apps from
sources other than Apple's App Store. Nor does it give you direct access to the
filesystem. Therefore, you need to be a registered iOS developer to install
your own apps during development; information about how to enroll in the iOS
Developer Program can be found on Apple's website\cite{iOSDevProgram}.

Once enrolled, you'll need to sign in to your user account in the Xcode IDE;
this is currently done from the ``Accounts'' tab in Xcode's preferences.
The Xcode project is available as \mbox{\texttt{platforms/ios/Power 
Supply.xcodeproj}} under the root directory of the project; once the project
is opened, you may install and run the app on an iPhone by performing the
following steps:

\begin{enumerate}
  \item Connect your iPhone to your Mac through a cable.
  \item Select your iPhone as the target device from the drop-down menu near
        the top-left of the project window.
  \item Click the ``Play'' icon or select Product $\rightarrow$ Run from the
        menu bar.
\end{enumerate}

These details may change in the future. For the full and current details,
please see the Apple Developer Support Center\cite{appleSupport}.

As a practical side note, Espen Skarsbø Olsen of Helgelandskraft set up a
company account for Helgelandskraft with the iOS Developer Program in October
of 2013.
