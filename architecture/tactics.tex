\subsection{Architectural Tactics}
In this section we will list the most important quality attributes as well as plan the tactis to 
ensure that we maintain the quality.

{\bf What is architectural tactics? }
Architectual tactics are design decisions that influence the control of a quality attribute reponse. 
Each tactic is a design option for the architect. 
Source: \url{http://www.ece.ubc.ca/~matei/EECE417/BASS/ch05lev1sec1.html}

\subsubsection{Modifiability}
To achieve good modifiability for our code, we will use well known patterns, like MVP and create 
independent modules. We will make sure each module only has one task and that each module works 
largely independent from the other modules in the game in order to achieve high cohesion and loose 
coupling. Loose coupling enables changes to one class without affecting others. High cohesion makes 
it easier to understand what each module does. This let's us minimize the "ripple effect" and 
increases modifiability.

\subsubsection{Performance}
The most critical resource for the performance of this project is CPU power. JavaScript is not very 
efficient, so making the code efficient is very important. A tactic we will be using to achieve this 
is to control the event sampling frequency. A typical event is a change to the game state which 
requires the game to be repainted. By only repainting the game at fixed intervals and only when a 
notable change has occured, we can improve the performance of the code.