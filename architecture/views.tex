\subsection{Architectural Views}

\subsubsection{Logical View} % which is the object model of the design, end-user functionality (Class diagrams)

\includegraphics[width=\textwidth]{pictures/class_diagram}

-->> The screen class will probably become connected to more classes as the GUI is planned. <<--

\subsubsection{Process View} % captures the concurrency and synchronization aspects of the design, itegrators, performance, scalability. Takes into account some non-functional requirements, like performance and reliability.
\includegraphics[width=\textwidth]{pictures/process_view_screen_flow}

The main menu screen is the entry point for the game. From there you can go to the how to play screen, 
to read about how the game works. You can go to the Level Select screen to start a new level, or load the 
most recent level and continue playing from where you last left of. While playing you can exit the current 
level and return to the main menu.

\subsubsection{Physical View} % describes the mappings of the software onto the hardware, and reflects its distributed aspect, system engineers, topology, communications

\includegraphics[width=\textwidth]{pictures/physical_view}

The physical architecture of our product is seperated into 3 layers. Platform specific issues are handled 
using PhoneGap as a cross platform framework between the application and the mobile OS.

{\bf Layer 1 - Native platform}
The first layer is the native operative system that runs on the device. This layer handles touchscreen 
functionality, rendering to screen, data input/output and other platform specific features.

{\bf Layer 2 - Cross platform framework}
The cross platform framework is called PhoneGap. The purpose of this layer is to allow the development 
team to write portable code that can run on both Android and iOS, without having to develop for both 
Android and iOS separately in 2 different programming languages. PhoneGap uses browser functionality on 
each platform to run HTML5, CSS and JavaScript code and provides access to platform functionality like 
camera and touchscreen.

{\bf Layer 3 - Mobile Application}
The third layer is where the game is implemented using the MVC pattern and Backbone's MVP to help with
the implementation.


\subsubsection{Development View}
The development view is separated into hierarchically into layers. The first layer is the game state 
and logic layer. The Models store the current state of the game which gets updated by the game loop 
and input from the player. The second layer decides how the game state is to be presented to the 
player, this layer also receives user input from layer three which updates the game state. The third 
layer is the HTML code which handles the layout of the user interface and sends user input to the
second layer.

\includegraphics[width=\textwidth]{pictures/development_view}
