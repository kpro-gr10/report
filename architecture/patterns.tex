\subsection{Architectural Patterns}

A common architectual pattern for video games, is to use a while loop, that runs for the duration of the game. Often using some timing mechanism to update the game at a steady rate. This pattern has it's drawbacks for purpose of this project. It is designed to redraw the screen every frame. On a mobile platform this would waste battery when no notable change has occured which needs to be redrawn. It can also be demanding on the hardware and can run badly on slower hardware. Therefore this project will be using the Model-View-Controller pattern using the JavaScript library Backbone.

The screen works as the view, and is notified whenever changes occur to the game models. The screen collects these events and schedules a repaint. The repaint does not happen immediately after the screen receives an event, but after a certain time, to prevent unnecessarily large spikes in repaints when a bunch of events happen at the same time, which is likely due to the nature of the game. This delaying of repaints is necessary because repaints are slow.

There are 2 singleton objects in the game. One is an Image Library. This object is responsible for loading and storing the sprites the game will draw to the screen. This will allow us to only read and store each sprite one time, saving memory and improving performance. The second singleton object is the Object Pool. This object creates game objects, and stores them while they are not placed on the game map. This object will create all the objects when the game loads, so that they do not have to be created later, further improving performance. Both these singletons are commonly used in video games.
