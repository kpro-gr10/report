\subsection{Architectural Drivers}
We are developing a game related to the power industry for Helgelandskraft. Helgelandskraft wants the game to be developed for Android and iOS. When the project is complete the product will be in the hands of Helgelandskraft and we will no longer be able to work on it. This means we have to use an architecture that Helgelandskraft can continue to develop with.

\subsubsection*{Cross Platform}
Helgelandskraft wants the product to be developed for Android and iOS. We had the choice between developing nativly for each platform. This would force us to develop in both Java and Objective-C in parallel or we would have to use a cross platform framework. We went for using a cross platform framework called Phonegap. However this can heavily impact the performance of the application, so we have to use an architecture that leverages this.

\subsubsection*{Developer Inexperience}
Our choice of the cross platform framework Phonegap comes with the drawback that nobody on the project group are very experienced with JavaScript. Which means we have to make the system fairly simple.

\subsubsection*{Projected Lifetime}
The customer have to be able to continue to provide support for the game, which means that there is no end-time for the project from the customers point of view while the student project ends November 21. This means that the product has to be easy to maintain after the end of the project.
