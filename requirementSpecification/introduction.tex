This section presents all the functional requirements, as well as the quality requirements.

{\bf Functional requirements:} State what the system must do, how it must behave or react 
to run-time stimuli. \\
{\bf Quality attribute requirements:} Annotate (quantify) functional requirement, e.g. how fast 
the function must be performed, how resilient it must be to erroneous input, how easy the 
function is to learn, etc.

{\bf The Priority Scale}:
\begin{itemize}
	\item {\bf Critical:} When a requirement is critical, it means that if the requirement is
	not fullfilled, the game will not be able to work properly. These requirements are of the
	type "need to have".
	\item {\bf High:} When a requirement is highly prioritized, it is important that it will
	be fullfilled in order to make the game work properly, but it will still work without it.
	These requirements are of the type "need to have".
	\item {\bf Medium: } In order to make the game fun and playable it is important that these are 
	fullfilled. This is more like a "nice to have" than "need to have".
	\item {\bf Low:} All the requirements that have a low priority are requirements that are "nice to have",
	but the game will work fine without them. If the project don't have time to implement all
	requirements, the requirements with low priority are the first to go.
\end{itemize}

{\bf The Complexity Scale:}
\begin{itemize}
	\item {\bf High:} A requirement are set with high complexity if the requirement takes a long 
	time to implement, are dependent on other requirements and/or if the developers do not know
	how to solve the implementation of the requirement.
	\item {\bf Medium:} A requirement are set with medium complexity if the requirement takes 
	some effort to implement, and is dependent on other requirements.
	\item {\bf Low:} A requirement are set with low complexity if the requirement do not take
	a long time to implement, the solution of the implementation is known and/or the requirement
	is not dependent on many other requirements.
\end{itemize}