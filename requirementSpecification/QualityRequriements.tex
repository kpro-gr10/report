\subsection{Quality Requirements}

In this section we will describe the quality requirements for the game and
will briefly describe all requirements under each quality requirement.
Each requirement will be described with a quality attribute scenario that looks like this:

\begin{itemize}
	\item {\bf Source of stimulus:} Human, computer, or other actor that generates the stimulus
	\item {\bf Stimulus:} condition that needs to be considered when it arrives at a system
	\item {\bf Environment:} the stimulus occurs within certain conditions
	\item {\bf Artifact:} some artifact is stimulated 
	\item {\bf Response:} the activity undertaken after the arrival of the stimulus
	\item {\bf Response measure:} when the response occur, it should be measured in some fashion so that the requirement can be tested.
\end{itemize}

\begin{figure}[!hr]
	\includegraphics{pictures/qualityAttribute.jpg}
\end{figure}

{\bf What is a quality attribute?:}

{\bf Priority of quality attributes:} It is not possible to meet all quality requirements because some 
decisions will affect the other. This will result in a priority of the quality attributes that we need 
to meet and some others that are not that important. We have picked two main quality attributes: 
modifiability and performance.

\begin{itemize}
	\item {\bf Primary quality attribute: } Modifiability
	\item {\bf Secondary quality attributes: } Performance
\end{itemize}


\subsubsection{Modifiability}

{\bf M1: Add new elements}

{\bf M2: Increase game difficulty}


\subsubsection{Performance}

{\bf P1: Rendering (Repaint)}

{\bf P2: Computational complexity}

\subsubsection{Usability}

{}

\subsubsection{Other Quality attributes}

{\bf Testability:}

{\bf Compitability:}