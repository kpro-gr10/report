\subsection{Usability Test}

A usability test will be performed in sprint 3 to discover errors and areas of improvement. First we have a look at what usability is.

\subsubsection{What is usability}

Usability can be defined as:

\begin{quote}
The effectiveness, efficiency, and satisfaction with which specified users 
achieve specified goals in particular environments. (ISO 9241-11)
\end{quote}

\paragraph{Effectiveness}

The easiest way to measure effectiveness is as the number of tasks completed without assistance. Effectiveness provides a quantitative measure of the degree to which users are able to perform the tasks that the product will offer. It van be measures as the number of tasks the user was able to perform on their own in a usability test. Problems experienced by the user will also be reported, because it tells us about aspects with the user interface that redused applicability.

Measuring effectiveness will be important to our product because the users of the game will not get any training in how the game works, they need to work this out on their own by reading the game instruction and trying the game. If they find it hard to perform the tasks in the game, they will probably not return to play it.

\paragraph{Efficiency}

Efficiency is about the resources used or time spent to carry out a task. It can be measured as "time spent per task".
By measuring how long each task will take you get an idea of the system's efficiency.

It is important to us to find out if users can perform basic game tasks efficiently. Because if basic tasks, that is to be performed over and over again, are intakes long time to carry out, the game experience will deteriorate.

\paragraph{Satisfaction}

Subjective satisfaction says something about the user's subjective experience and assessment of the product. Satisfaction can be measured both quantitatively and qualitatively. The system usability scale (SUS) is a form of quantitative measure, where users rate various aspects of the product on a scale from 1 to 5. Qualitatively subjective satisfaction can be measured by interviewing the subjects after the test. Much can also be interpreted on the basis of what they say during the test. It is important to have some qualitative measure in addition to the quantitative, because the questions in the SUS form is general and may not cover all aspects of our product.

This is important to get insight in what opinions the user has about the product. What is working, what is not working and what could be improved.

\paragraph{Specified users in particular environments}

It is important to identify relevant user groups, and make sure that the testers is representative of the user groups. If the system is to be used in a particular environment, and if possible, this environment should be tried recreated, for the test. In our case, the environment is not of great importance, as the game can be played practically anywhere a mobile phone can be.

\subsubsection{Planning the usability test}

\paragraph{How far is the game from completion?}

\paragraph{Who are the players?}

\paragraph{How many players should be invited?}

Reference to plot.

\paragraph{How will players be recruited?}

\paragraph{What should be tested?}

Playable

Understand the concept, how to play, the goal, how not to lose, 

how to earn money, from the process of building power plants, serving buildings and collecting money.
how to keep the health bar from decreasing, serve all houses.

Discover issues, problems, errors, vaguenesses, missing functionality, areas of improvement

The player will not be given instructions or scenarios explaining what to do, because we want the player find things out for himself, just like it would be if the player had just downloaded the app. We want to examine if the player manages to play the game without being told how to play, beyond the information the game instructions give. 

\paragraph{How will the test be executed?}

The player will get a brief oral explanation of the elements and goal of the game, and will be presented with the game instructions that are found in the game.

The player will then start a new game and will be encouraged to think aloud. 

- Explanations to why certain actions are chosen, the first time they're executed.
- Questions the player might have. For example "I want to be able to ..., but don't know how to do it". These questions will not be answered unless the player can not proceed in the game without help.
- Report on problems that arises.
- Report on bugs or errors that occur.
- Report on what was difficult to achieve.

\paragraph{Observation form}

During the execution of the test one of the team members will document all problems, bugs or errors the player encounters.


Map

Hud

Building

Buildings

Navigation

Upgrading

Money

Health

Other


\paragraph{Follow-up questions}

After the player has played the game a small interview will take place, where we present the player with a list of pre-made questions. The player will also be encouraged to give feedback on matters, he or she felt the questions did not cover.

\paragraph{System Usability Scale}

Finally the player will be asked to fill out a SUS form. (Ha med at man kan legge til begrunnelse?)

Questions:

(From web page)

1. What age range do you think this game is suitable for?

2. Complexity: very simple, average, very complex

3. Game Instructions/Rules: very simple, average, very complex

4. Luck vs. Skill: pure luck, half luck, half skill, all skill

5. Uniqueness / Game Mechanics (How different was this game from other games?): Not much different, Very different

6. Playing time (Was the game too short, too long or just right?): Too short, Just right, Too long

7. Appearance (How much did you like the graphics/illustrations?): Did not like, Loved

8. Game Idea (Concept) or Theme: Boring or weak, OK, Terrific

9. Interest (How much did you like this game?): Hated it, It was OK, Loved it

10. Repeat Play (How often will you play this game?): never again, now and then, a lot

11. Game Options:  Are there not enough options for what you can do, too many options (too many choices) or just the right amount? not enough, just right, too many

12. Gamemap size: Was the gamemap too small, too big, or just the right size? too small, just right, too big

13. Game elemenst (size): Were the elements too small, too big, or just the right size? too small, just right, too big

14. Text size: Was the text in the game, too small or just right? too small, just right, too big

15. Do you have any specific complaints, or precise suggestions that you feel would make the game better, especially in terms of the movement of the pieces, the balance and the object of the game?

(From SUS)

1. I think that I would like to play this game frequently.

2. I found the game unnecessarily complex.

3. I thought the game was easy to understand.

4. I found the various functions in this game were well integrated.

5. I thought there was too much inconsistency in this game.

6. I would imagine that most people would learn to play this game very quickly.

7. I found the game very cumbersome(tungvint) to play.

8. I felt very confident playing the game.

9. I needed to learn a lot of things before I could start playing the game.




