\subsection{Usability Test}

A usability test will be performed in sprint 3 to discover errors and areas of improvement. First we have a look at what usability is.

\subsubsection{What is usability}

Usability can be defined as:

\begin{quote}
The effectiveness, efficiency, and satisfaction with which specified users 
achieve specified goals in particular environments. (ISO 9241-11)
\end{quote}

\paragraph{Effectiveness}

The easiest way to measure effectiveness is as the number of tasks completed without assistance. Effectiveness provides a quantitative measure of the degree to which users are able to perform the tasks that the product will offer. It van be measures as the number of tasks the user was able to perform on their own in a usability test. Problems experienced by the user will also be reported, because it tells us about aspects with the user interface that redused applicability.

Measuring effectiveness will be important to our product because the users of the game will not get any training in how the game works, they need to work this out on their own by reading the game instruction and trying the game. If they find it hard to perform the tasks in the game, they will probably not return to play it.

\paragraph{Efficiency}

Efficiency is about the resources used or time spent to carry out a task. It can be measured as "time spent per task".
By measuring how long each task will take you get an idea of the system's efficiency.

It is important to us to find out if users can perform basic game tasks efficiently. Because if basic tasks, that is to be performed over and over again, are intakes long time to carry out, the game experience will deteriorate.

\paragraph{Satisfaction}

Subjective satisfaction says something about the user's subjective experience and assessment of the product. Satisfaction can be measured both quantitatively and qualitatively. The system usability scale (SUS) is a form of quantitative measure, where users rate various aspects of the product on a scale from 1 to 5. Qualitatively subjective satisfaction can be measured by interviewing the subjects after the test. Much can also be interpreted on the basis of what they say during the test. It is important to have some qualitative measure in addition to the quantitative, because the questions in the SUS form is general and may not cover all aspects of our product.

This is important to get insight in what opinions the user has about the product. What is working, what is not working and what could be improved.

\paragraph{Specified users in particular environments}

It is important to identify relevant user groups, and make sure that the testers is representative of the user groups. If the system is to be used in a particular environment, and if possible, this environment should be tried recreated, for the test. In our case, the environment is not of great importance, as the game can be played practically anywhere a mobile phone can be.

\subsubsection{Planning the usability test}

How far is your game from completion?

Who are your players?

How many players should you invite?

How will you recruit your players?

What do you want to test?



scenarios

observasjonsskjema

sus








1. Utvikling av målsettinger og hypoteser for testen og testplan 
	Hvem er kunde for brukbarhetstesten? 
	Hva er hensikten med testen? 
	Hva skal resultatene brukes til? 
	Gjør eksplisitt hvilke ressurser som trengs 
	Når skal testen gjennomføres? 
	Hvem er ansvarlig? 
	Fungerer som kommunikasjon mellom de som er involvert.


2. Skaffe brukere gjennom tilfeldig eller stratifisert utvalg. 
	Eksempel på brukerkarakteristikker: 
		Personlig historie: 
			Alder, kjønn, holdninger til typen av produkt, ”hendthet”, læringsstil, holdning til teknologi 
		Utdanningshistorie: 
			Høyest oppnådde grad, Fag 
		Erfaring med IKT: 
			(hvor lenge, hvor ofte, hvilken type operativsystem) 
		Produkterfaring 
			Tid brukt på produktet, frekvens, hvilken type oppgaver, hvilken type 
			(bruker de "ditt" produkt)? 
		Jobbhistorie 
			Nåtidig og tidligere jobb, ansvar for hva, hvor lenge i nåtidig stilling etc.


 	Antall brukere avhenger av flere faktorer: 
		Hvor høy grad av objektivitet som er nødvendig 
		Ressurser man har til rådighet for å utføre testen 
		Tilgjengelighet av brukere 
		Varigheten av testen 
		Kompleksiteten av brukergrensesnittet 
		Normalt vil antallet for en brukbarhetstest ligge mellom 5 og 8 

3. Forberede materiale og kontekst 
	Forbered testen i detalj. 
	Hva skal testes (programvare, papirprototyp,,)? 
	I hvilken omgivelse skal den testes (kontor, simulert sykehus, kafe,,)? 
	Hvilke oppgaver skal brukerne løse?

	Utvikle et scenario for testen 
		Lag så realistiske og komplette scenarioer som mulig. Bruk gjerne case studier, oppgaveanalyser eller faktiske observasjoner for å få scenarioet så realistisk som mulig. 
		Lag scenarioet slik at rekkefølgen på oppgaven tilsvarer det den ville gjort i virkeligheten. 
		Match scenarioet i forhold til brukerne: Enkle scenarier til noviser, mer komplekse for eksperter. 
		Unngå jargon 
		Unngå å gi "hint" til oppgavens løsning, gjennom måten man presenterer oppgaven på. Unngå å bruke ord som tilsvarer navnet på den funksjonen du vil at de skal bruke 
		Presenter oppgaven i målrettet form med et enkelt språk (presenter målet med oppgaven, ikke enkeltstegene)
		Sett opp brukbarhetslaboratoriet

4. Velge forsøksleder 
	Hvem bør være forsøksleder? 
	Intern human factors - spesialist (akseptabel objektivitet, akseptabel kunnskap om produktet) 
	Marketing spesialister / teknikere (god kjennskap til produktet, men kan ha lav grad av objektivitet) 
	Ekstern konsulent (god objektivitet / profesjonalitet, kan ha lav kjennskap til produktet) 

5. Pilot test 
	Å gjennomføre en pilottest er viktig for å oppdage svakheter ved metodikken og test-designet 
	Gir testteamet anledning til å øve seg 

6. Utføre testen: 
	10 retningslinjer:

	1. Introduser deg selv 
	2. Beskriv hensikten med testen 
	3. Fortell deltakerene at de kan avbryte når de vil 
	4. Beskriv utstyret i rommet og begrensningene til prototypen 
	5. Lær bort hvordan man tenker høyt 
	6. Forklar at du ikke kan tilby hjelp under testen 
	7. Beskriv oppgaven og introduser produktet 
	8. Spør om det er noe de lurer på og kjør testen 
	9. Avslutt testen med å la brukeren uttale seg før du samler evnt. løse tråder 
	10.Bruk resultatene

7. Omforming av data til funn og anbefalinger 
	Identifiser feilhandlinger og problemer ved logging av sammenbrudd 
	"Real time" score-ing 
	Videoanalyser 
	Forsøk å komme til bunns i hva problemene skyldes (de mer bakenforliggende problemene) 
	Prioriter funnene (fordi funnene kan være interavhengige, bør man fokusere på de viktigste funnene først) 
	Utvikle løsningsforslag 
	Indiker hvor man trenger ytterligere forskning 
	Produser rapport 



