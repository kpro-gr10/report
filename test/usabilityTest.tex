\subsection{Usability Test}

A usability test will be performed in sprint 3 to discover errors and areas of improvement. First we have a look at what usability is.

\subsubsection{What is usability}

Usability can be defined as:

\begin{quote}
The effectiveness, efficiency, and satisfaction with which specified users 
achieve specified goals in particular environments. (ISO 9241-11)
\end{quote}

\paragraph{Effectiveness}

The easiest way to measure effectiveness is as the number of tasks completed without assistance. Effectiveness provides a quantitative measure of the degree to which users are able to perform the tasks that the product will offer. It van be measures as the number of tasks the user was able to perform on their own in a usability test. Problems experienced by the user will also be reported, because it tells us about aspects with the user interface that redused applicability.

Measuring effectiveness will be important to our product because the users of the game will not get any training in how the game works, they need to work this out on their own by reading the game instruction and trying the game. If they find it hard to perform the tasks in the game, they will probably not return to play it.

\paragraph{Efficiency}

Efficiency is about the resources used or time spent to carry out a task. It can be measured as "time spent per task".
By measuring how long each task will take you get an idea of the system's efficiency.

It is important to us to find out if users can perform basic game tasks efficiently. Because if basic tasks, that is to be performed over and over again, are intakes long time to carry out, the game experience will deteriorate.

\paragraph{Satisfaction}

Subjective satisfaction says something about the user's subjective experience and assessment of the product. Satisfaction can be measured both quantitatively and qualitatively. The system usability scale (SUS) is a form of quantitative measure, where users rate various aspects of the product on a scale from 1 to 5. Qualitatively subjective satisfaction can be measured by interviewing the subjects after the test. Much can also be interpreted on the basis of what they say during the test. It is important to have some qualitative measure in addition to the quantitative, because the questions in the SUS form is general and may not cover all aspects of our product.

This is important to get insight in what opinions the user has about the product. What is working, what is not working and what could be improved.

\paragraph{Specified users in particular environments}

It is important to identify relevant user groups, and make sure that the testers is representative of the user groups. If the system is to be used in a particular environment, and if possible, this environment should be tried recreated, for the test. In our case, the environment is not of great importance, as the game can be played practically anywhere a mobile phone can be.

\subsubsection{Planning the usability test}\mbox{}\\

\paragraph{How far is the game from completion?}\mbox{}\\



\paragraph{Who are the players?}\mbox{}\\

The target group for this product is people that own a smartphone and have an interest in apps and playing games. We also hope to reach people who are the ones paying for the electricity in their household. 

\paragraph{How many players should be invited?}\mbox{}\\

"KILDE" shows that with 15 testers the number of usability problems found reaches 15. It also shows that with more than 5 testers one learns less and less when introducing additional testers. Taking into account the little time we have at hand, we feel 5 players will be sufficient for this usability test-

\paragraph{How will players be recruited?}\mbox{}\\




\paragraph{What should be tested?}\mbox{}\\

We want to examine if people understand the concept of the game, how to play it, the goal of the game, 
and how not to lose. Is it intuitive for the player to understand how one earns money? From the process 
of building power plants, serving buildings and collecting money. Is it intuitive how one keeps 
the health bar from decreasing? That is by serving all the houses with power.

We want to discover issues, problems, errors, vaguenesses, missing functionality and other areas 
of improvement. We want to inquire whether the testers find the game fun to play, or if it could 
be more fun with certain improvements.

The player will not be given instructions or scenarios explaining what to do, because we want 
the player to find things out for himself, just like it would be if the player had just 
downloaded the app. We want to examine if the player manages to play the game without being 
told how to play, beyond a small introduction and the information the game instructions provide. 

\paragraph{How will the test be executed?}\mbox{}\\

The player will get a brief oral explanation of the elements and goal of the game, 
and will be presented with the game instructions that are found in the game.

The player will then start a new game and will be encouraged to think aloud. 

These thoughts may regard:

\begin{itemize}
  \item Explanations to why certain actions are chosen, the first time they're executed.
  \item Questions the player might have. Questions will not be answered unless the player can not proceed in the game without help, they but can help us discover areas that needs improvement.
  \item Problems that arises. For example "I want to be able to ..., but don't know how to do it". 
  \item Bugs or errors that occur.
  \item What was difficult to achieve.
\end{itemize}

\paragraph{Observation form}\mbox{}\\

During the execution of the test one of the team members is the observator, and will document all problems, bugs or errors the player encounters. The observator reports what the problem is, what causes the problem and what will be a possible solution to the problem. The observation form can be found in Appendix XX.

\paragraph{Follow-up questions}\mbox{}\\

After the player has played the game, a small interview will take place, where the player is presented with potential questions the observator has found during the test, as well as two pre-made questions. The player will also be encouraged to give feedback on matters, he or she felt the questions did not cover.

1. Do you have any specific complaints, or precise suggestions that you feel would make the game better?

2. I found the game very cumbersome to play.

\paragraph{Usability Survey}\mbox{}\\

Finally the player will be asked to fill out a survey, where he is asked to rate certain aspects of the game on a scale from 1 to 5. There are in all 18 questions. The first 8 questions are fetched from the System Usability Scale and adapted to our problem, whereas the 10 following questions are more specific to games, and the scale of each question has its own specified rating. The form with the questions can be found in Apendix XX.

