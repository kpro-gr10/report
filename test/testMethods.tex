\section{Test Methods}

\paragraph{Black box testing}

Black box testing examines that the functionality of the application works according to the requirements. To carry out a black box test knowledge of internal structure of the application is not necessary. Knowledge of what the application is supposed to do is required, but no how it does it. Hence the software is viewed as a black box, where you can not see the inside, but where input is tested against output to see if the application acts as expected. The strong point of black box testing is detecting unimplemented parts of the specification or missing requirements.

\paragraph{White box testing}

White box testing on the other hand examines the internal structure of the application. In the case of white box testing the tester needs knowledge of \emph{how} the application does what it is supposed to do as well as knowledge about programming. The strong point of white box testing is its ability to uncover errors or problems, and where they are located. Black box testing can detect errors as well, but it can be harder to locate the origin of the errors.

Below, different methods of testing are described briefly.

\begin{description}
  \item[Unit Test] Individual units of source code are tested independently for correctness.
  \item[Functional Test] A system is tested by feeding it input and examining the output against the requirements.
  \item[Usability Test] Examining the products ease of use by testing it on users.
  \item[Compatibility Test] Evaluating the products compatibility with other system software etc.
  \item[Integration Test] Individual software modules are combined and tested as a group.
  \item[Performance Testing] Determine how the system performs in terms of the performance requirements.
\end{description}