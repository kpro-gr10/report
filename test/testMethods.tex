\subsection{Test Methods}

\paragraph{Black box testing}

Black box testing examines that the functionality of the application works according to the requirements. To carry out a black box test knowledge of internal structure of the application is not neccessary. Knowledge of what the application is supposed to do is required, but no how it does it. Hence the software is viewed as a black box, where you can not see the inside, but where input is tested against output to see if the application acts as expected. The strong point of black box testing is detecting unimplemented parts of the specification or missing requirements.

\paragraph{White box testing}

White box testing on the other hand examines the internal structure of the application. In the case of white box testing the tester needs knowledge of \emph{how} the application does what it is supposed to do as well as knowledge about programming. The strong point of white box testing is its ability to uncover errors or problems, and where they are located. Black box testing can detect errors as well, but it can be harder to locate the origin of the errors.

\begin{description}
  \item[Unit Test] The first item
  \item[Integration Test] The second item
  \item[System Test] The third etc
  \item[Usability Test]
  \item[Acceptance Test]
\end{description}