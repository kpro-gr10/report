\section{Testing Approach}

\subsection{What will be tested}

	Unit testing is a form of white box testing and is a very important part of every software development project. If a unit test fails it is relatively easy to detect the error, and they can be detected early in process. If, or rather when, code is changed, the impact the change has on other parts of the code can easily be determined through unit testing. Due to the limited resources
	the group possesses in both time and people, unit tests will only be written if there is time. 

	The main focus will lie on functionality testing. Functionality testing is a type of black box testing and it will be performed on an iPhone and an Android phone running the game app.

	Usability testing will also be an important part of the project, considering that if users find the application difficult to use or hard to understand, they can very well decide not to use it, as there are thousands of other games out there.

	During one of the last sprints a compatibility test will be carried out, to get an overview of how the application looks on different smart phones.

	A user acceptance test will be performed at the end of the project to let the customer test the final product and
	approve or disapprove it.

\subsubsection{What will not be tested}

	Integration testing: Because the application does not use servers or databases, an integration test would be superfluous. Testing of the application as a whole will be done through the various functional tests.

	Performance testing: Tests if the system working according to the performance demands from the customer. YESNO??????