
intro

meets the requirements that guided its design and development,
works as expected,
can be implemented with the same characteristics,
and satisfies the needs of stakeholders

\subsection{Test Methods}

\paragraph{Black box testing}

Black box testing examines that the functionality of the application works according to the requirements. To carry out a black box test knowledge of internal structure of the application is not neccessary. Knowledge of what the application is supposed to do is required, but no how it does it. Hence the software is viewed as a black box, where you can not see the inside, but where input is tested against output to see if the application acts as expected. The strong point of black box testing is detecting unimplemented parts of the specification or missing requirements.

\paragraph{White box testing}

White box testing on the other hand examines the internal structure of the application. In the case of white box testing the tester needs knowledge of \emph{how} the application does what it is supposed to do as well as knowledge about programming. The strong point of white box testing is its ability to uncover errors or problems, and were they are located. Black box testing can detect errors as well, but it can be harder to locate the origin of the errors.

\begin{description}
  \item[Unit Test] The first item
  \item[Integration Test] The second item
  \item[System Test] The third etc
  \item[Usability Test]
  \item[Acceptance Test]
\end{description}


\subsection{Testing Approach}

\subsubsection{What will be tested}

Unit testing is a very important part of every software development project. If a unit test fails it is relatively easy to detect the error, and they can be detected early in process. If, or rather when, code is changed, the impact the change has on other parts of the code can easily be determined through unit testing. Due to the limited resources
the group possesses in both time and people, unit tests will only be written if there is time. 

The main focus will lie on functionality testing. Functionality testing is a type of black box testing and it will be performed on an iPhone and an Android phone running the game app.

Usability testing will also be an imortant part of the project, considering that if users find the app diffucult to use or hard to understand, they can well decide not to use it, as there are thousands of other games out there.

A user acceptance test will be performed at the end of the project to let the customer test the final product and
approve or disapprove it.

\subsubsection{What will not be tested}


Integration testing: Because the application does not use servers or databases, an integration test would be superfluous. Testing of the application as a whole will be done through the various functional tests.

Performance testing: Tests if the system working according to the performance demands from the customer. YESNO??????

\subsection{Test Case Overview}

Test cases have been developed for the testing of the fuction of the application. Each test case has been given an identification FT-XX, where FT stands for Functional Test and XX is the number of the test.

Below is the template for the test cases.

\begin{table}[H]
\centering
	\begin{tabular}{ l | p{8cm} }
		\hline
		{\bf Item} & {\bf Description} \\ \hline
		Name & The name of the test \\ 
		Test Id & Identifier of the test \\ 
		Feature to be tested & States the functionality to be tested \\ 
		Requirement & The corresponding requirements that are tested \\ 
		Pre-conditions & Conditions that are fulfilled and can be assumed is working before the test is executed. \\ 
		Steps of Execution & The steps of how the test is to be carried out. \\ 
		Expected result & The expected outcomes of the steps carried out above. \\ 
	\end{tabular}
	\caption{Test case template}
\end{table}

\subsection{Test Cases}

\begin{table}[H]
\centering
	\begin{tabular}{ l | p{8cm} }
		\hline
		{\bf Item} & {\bf Description} \\ \hline
		Name & Map navigation \\ 
		Test Id & FT-01 \\ 
		Feature to be tested & That user should be able to see a small part of the game map and be able to navigate around it.\\ 
		Requirement & FR6.1 and FR6.2 \\ 
		Pre-conditions & None \\ 
		Steps of Execution & 1. Drag a finger over the screen. \\
		& 2. Try to navigate outside the game map. \\
		Expected result & 1. The screen will move across the map in the opposite direction of the finger.\\ 
		& 2. Nothing happens. \\
	\end{tabular}
	\caption{Functional test 1}
\end{table}

\begin{table}[H]
\centering
	\begin{tabular}{ l | p{8cm} }
		\hline
		{\bf Item} & {\bf Description} \\ \hline
		Name & Appearance of buildings \\ 
		Test Id & FT-02 \\ 
		Feature to be tested & That buildings appear around the map at arbitrary intervals and locations. \\ 
		Requirement & FR2.1 \\ 
		Pre-conditions & FT-01 \\ 
		Steps of Execution & 1. Observe that buildings appear around the map at arbitrary intervals and locations.\\ 
		Expected result & 1. Buildings appear around the map at arbitrary intervals and locations at a satisfying rate.\\ 
	\end{tabular}
	\caption{Functional test 2}
\end{table}

\begin{table}[H]
\centering
	\begin{tabular}{ l | p{8cm} }
		\hline
		{\bf Item} & {\bf Description} \\ \hline
		Name & Zooming \\ 
		Test Id & FT-03 \\ 
		Feature to be tested & The shifting between zoomed-out and zoomed-in screens. \\ 
		Requirement & FR6.4 \\ 
		Pre-conditions & FT-01 \\ 
		Steps of Execution & 1. Double tap the screen in zoomed-in mode.\\ 
		& 2. Double tap the screen in zoomed-out mode. \\
		Expected result & 1. The screen will show the whole game map. \\
		& 2. The game will show the part of the game map where the player tapped the screen. \\
	\end{tabular}
	\caption{Functional test 3}
\end{table}

\begin{table}[H]
\centering
	\begin{tabular}{ l | p{8cm} }
		\hline
		{\bf Item} & {\bf Description} \\ \hline
		Name & Main Menu \\ 
		Test Id & FT-04 \\ 
		Feature to be tested & That the buttons in the main menu directs to the correct pages. \\ 
		Requirement & FR1.1, FR6.7 ?Highscore?, ?Start Game? \\ 
		Pre-conditions & FT-01 \\ 
		Steps of Execution & 1. Tap the 'New Game' button. \\
		& 2. Tap the 'Instructions' button. \\
		& 3. Tap the 'Highscore' button. \\
		Expected result & 1. A new game is started. \\
		& 2. The page showing the game instructions is shown. \\
		& 3. The page showing the high score is shown. \\
	\end{tabular}
	\caption{Functional test 4}
\end{table}

\begin{table}[H]
\centering
	\begin{tabular}{ l | p{8cm} }
		\hline
		{\bf Item} & {\bf Description} \\ \hline
		Name & Build Power Plants \\ 
		Test Id & FT-05 \\ 
		Feature to be tested & That it is possible to buy and place power plants on the game map. \\ 
		Requirement & FR2.2, FR6.8 \\ 
		Pre-conditions & FT-01. First the player can afford a Power Plant, subsequently the player can not afford a Power Plant. \\
		Steps of Execution & 1. Drag the build menu into view. \\ 
		& 2. Tap the power plant icon. \\
		& 3. Tap somewhere on the map where a power plant can not be placed e.g. a building. \\
		& 4. Tap an emty area on the map. \\
		& 5. Tap 'OK'. \\
		& 6. Repeat, but tap 'Cancel'. \\
		Expected result & 2.1 If player has enough money the game is set to building mode. \\
		& 2.2 If player does not have enough money, information is displayed. \\
		& 3. Nothing happens. \\ 
		& 4. The player is aksed to either buy or cancel. \\
		& 5. The power plant is placed on the indicated spot and the players amount of money is reduced. \\
		& 6. Leave building mode. \\
	\end{tabular}
	\caption{Functional test 5}
\end{table}

\begin{table}[H]
\centering
	\begin{tabular}{ l | p{8cm} }
		\hline
		{\bf Item} & {\bf Description} \\ \hline
		Name & Build Power Lines \\ 
		Test Id & FT-06 \\ 
		Feature to be tested &  That it is possible to buy power cables and connect buildings and power plants. Proportional COST!! \\ 
		Requirement & FR2.3, FR5.2, FR6.8 \\ 
		Pre-conditions & FT-01 \\ 
		Steps of Execution & \\ 
		Expected result & \\ 
	\end{tabular}
	\caption{Functional test 6}
\end{table}















\begin{table}[H]
\centering
	\begin{tabular}{ l | p{8cm} }
		\hline
		{\bf Item} & {\bf Description} \\ \hline
		Name & Exit Game \\ 
		Test Id & FT-02 \\ 
		Feature to be tested & Test that the game is paused when exiting the game. \\ 
		Requirement & FR1.2 \\ 
		Pre-conditions & \\ 
		Steps of Execution & 1. Leave the game by tapping the home button on the phone. \\
		& 2. Open the game app. \\
		& 3. Tap 'Load Game'. \\
		Expected result & 3. The game is paused at the point where it was left. \\  
	\end{tabular}
	\caption{Functional test 2}
\end{table}

\begin{table}[H]
\centering
	\begin{tabular}{ l | p{8cm} }
		\hline
		{\bf Item} & {\bf Description} \\ \hline
		Name & Upgrade Power Plant\\ 
		Test Id & FT-07 \\ 
		Feature to be tested & That it is possible to upgrade a power plant. \\ 
		Requirement & FR2.4, FR5.3, FR6.3 \\ 
		Pre-conditions & First the player does not have enough money. Subsequently the player does have enough money. \\ 
		Steps of Execution & 1. Tap a power plant. \\ 
		& 2. Choose 'Upgrade Power Plant'. \\
		& 3. Choose "Upgrade" from the pop up. \\
		Expected result & 1. An information box about the cost of upgrading and the effects of upgrading is displayed. \\
		& 2. The system asks whether to 'Upgrade' or 'Cancel'. \\
		& 3. When not having enough money the system gives accordant feedback. \\
		& 3. When having enough money the system upgrades the power plant. \\
	\end{tabular}
	\caption{Functional test 7}
\end{table}

\begin{table}[H]
\centering
	\begin{tabular}{ l | p{8cm} }
		\hline
		{\bf Item} & {\bf Description} \\ \hline
		Name & Enter next level \\ 
		Test Id & FT-09 \\ 
		Feature to be tested & That when one completes a level one is able to continue to the next level. \\ 
		Requirement & FR4.1 \\ 
		Pre-conditions & \\ 
		Steps of Execution & \\ 
		Expected result & \\ 
	\end{tabular}
	\caption{Functional test 9}
\end{table}

\begin{table}[H]
\centering
	\begin{tabular}{ l | p{8cm} }
		\hline
		{\bf Item} & {\bf Description} \\ \hline
		Name & Game Over \\ 
		Test Id & FT-10 \\ 
		Feature to be tested & That the game is over when the health score reaches zero. \\ 
		Requirement & FR4.2 \\ 
		Pre-conditions & \\ 
		Steps of Execution & 1. Let several buildings not get power. \\ 
		& 2. Let the health score decrease to zero. \\
		& 3. Tap "Start new game". \\
		& 4. Tap "Back to main menu". \\
		Expected result & 1. The health score is decreasing and houses not supplied are red. \\
		& 2. The "Game Over" page is displayed. \\
		& 3. A new game is started. \\
		& 4. The main menu page is displayed. \\
	\end{tabular}
	\caption{Functional test 10}
\end{table}

\begin{table}[H]
\centering
	\begin{tabular}{ l | p{8cm} }
		\hline
		{\bf Item} & {\bf Description} \\ \hline
		Name & Collect money \\ 
		Test Id & FT-12 \\ 
		Feature to be tested & That it is possible to collect money from buildings connected to the power plants at regular intervals. \\
		Requirement & FR5.1 \\ 
		Pre-conditions & A building is ready to pay for power. \\ 
		Steps of Execution & 1. Tap the building. \\ 
		Expected result & 0. The building has a collect money sign. \\
		& 1. The players money is updated and the sign disappears. \\
	\end{tabular}
	\caption{Functional test 12}
\end{table}

\begin{table}[H]
\centering
	\begin{tabular}{ l | p{8cm} }
		\hline
		{\bf Item} & {\bf Description} \\ \hline
		Name & Tilting \\ 
		Test Id & FT-13 \\ 
		Feature to be tested & That the screen picture is not tiltet when the phone is tilted. \\ 
		Requirement & FR1.12 \\ 
		Pre-conditions & \\ 
		Steps of Execution & 1. Tilt the phone. \\ 
		Expected result & The screen picture stays in place. \\ 
	\end{tabular}
	\caption{Functional test 13}
\end{table}

\begin{table}[H]
\centering
	\begin{tabular}{ l | p{8cm} }
		\hline
		{\bf Item} & {\bf Description} \\ \hline
		Name & Building Mode \\ 
		Test Id & FT.14 \\ 
		Feature to be tested & That the game is set in "Building Mode" when building power plants or power cables. \\ 
		Requirement & FR6.9 \\ 
		Pre-conditions &  \\ 
		Steps of Execution &  \\ 
		Expected result & \\ 
	\end{tabular}
	\caption{Functional test 14}
\end{table}

\begin{table}[H]
\centering
	\begin{tabular}{ l | p{8cm} }
		\hline
		{\bf Item} & {\bf Description} \\ \hline
		Name & No Power and Health Bar \\ 
		Test Id & FT.15 \\ 
		Feature to be tested & \\ 
		Requirement & FR3.6, FR6.10 \\ 
		Pre-conditions &  \\ 
		Steps of Execution &  \\ 
		Expected result & \\ 
	\end{tabular}
	\caption{Functional test 15}
\end{table}

\begin{table}[H]
\centering
	\begin{tabular}{ l | p{8cm} }
		\hline
		{\bf Item} & {\bf Description} \\ \hline
		Name & Information on Building \\ 
		Test Id & FT-16 \\ 
		Feature to be tested &  \\ 
		Requirement & FR2.8 \\ 
		Pre-conditions & \\ 
		Steps of Execution & \\ 
		Expected result & \\ 
	\end{tabular}
	\caption{Functional test 16}
\end{table}


\subsection{Test Plan Schedule}

This section outlines the schedule of which sprints the test cases were executed, as seen in table XX. Some of the tests were executed in several sprints. The schedule shows the first sprint in which the test was performed. The results of the tests are avaiable in the Test section of each sprint.

\definecolor{lightgray}{gray}{0.9}

\begin{tabular}{| l | l | l |}
	\hline
	\rowcolor{lightgray}
	{\bf Test Case} & {\bf Sprint} \\ \hline
	FT-01 & \\ \hline
	FT-02 & \\ \hline
	FT-03 & \\ \hline
	FT-04 & \\ \hline
	FT-05 & \\ \hline
	FT-06 & \\ \hline
	FT-07 & \\ \hline
	FT-08 & \\ \hline
	FT-09 & \\ \hline
	FT-10 & \\ \hline
	FT-11 & \\ \hline
	FT-12 & \\ \hline
	FT-13 & \\ \hline
	FT-14 & \\ \hline
	FT-15 & \\ \hline
	FT-16 & \\ \hline
	FT-17 & \\ \hline
	FT-18 & \\ \hline
	FT-19 & \\ \hline
	FT-20 & \\
	\hline
\end{tabular}