\subsection{Quality Assurance}

This section describes the standards and routines that have been set by the group in order to achieve Quality Assurance. The aim is to ensure an overall quality of the project process and the outcome of the project. the consistency of all documentation and a common understanding that the whole process runs as smoothly as possible

\subsubsection{Internal Routines and Communication}

As many of the group members have a tight schedule the group will be flexible on when to work together so as to make sure that as many as possible of the group members can meet. No fixed meeting time has therefore been set, but 3-6 days a week will be the norm. The group will strive to work together as much as possible so that when problems arises or something is not clear, these can be dealt with in plenary. At the beginning of each meeting everyone will in turn give a brief update on what they have done, what problems they might have and what they are planning to do that day. The date, time and place for meetings will be updated in the groups Google Calendar. The group members will also be given tasks to work on outside of meetings. At the end of each working day, the group members will update their own part of the timesheet. The timesheet contains which tasks have been worked on, on a certain date and how much time was spent on each task. 

\subsubsection{External Routines and Communication}

\paragraph{Customer}

The customer is not based in Trondheim and consequently communication will mainly take place via e-mail and Skype. A customer contact from the group has been appointed to ensure unanimous communication outwards and to let the customer relate to the group as a whole as opposed to four individual people. The customer contact is the project manager. 

Customer meetings are held at the end of each sprint. This will mainly be on Fridays as long as both customer and the team can meet at the given date. The meetings take place using Skype, but if necessary an in-person meeting will be arranged. The customer contact sends a call for the meeting no later than 48 hours before the meeting is to take place. The customer and the group decide on a meeting day a few days before the meeting. Innhold, ref. template?. At the end of the meeting the customer will provide feedback on the phase results, and both the customer and the team will agree upon what the delivery of the next sprint should contain.

Before meetings one group member is appointed to write the minutes of the meeting. The minutes are to be written during the meeting and to be approved by all group members when the meeitng is over. They are to be sent for approval before 12 o'clock the following day. 

Questions and information for the customer should be sent by e-mail. All e-mail correspondence should be sent to all customer representatives, as well as all team members.

\paragraph{Advisor}

A weely meeting with the advisor will be held each Friday at 11.15. If the advisor or possibly group members are not able to meet at the given time, an alternative day and time will be set. The project manager will send a call for the meeting containg the time and place for the meeting, meeting agenda, status report and minutes from last advisor meeting. The call will be sent by e-mail by 12 o'clock the day before the meeting. Meeting minutes are to be written for each meeting. The advisor will have access to several of the groups documents on Google Drive, as well as the report and code on Git Hub. The documents include Project Plan, Risk Assessment, Requirement Specification and Timesheet. The advisor will have the possibility to follow the groups progress through these documents and will be able to read them at any time. If the advisor wants access to other important documents he will receive this. 

\subsubsection{Templates}

Templates for the regular documents, such as hour list and meeting minutes were developed. The templates ensured consistency. The templates can be found in Chapter 13.

\subsubsection{Response Time Lines}

In agreement with the customer these timeframes has been set for approval and response on documents and questions.

\begin{itemize}
	\item Approval of meeting minutes from customer meetings: 24h
	\item Feedback on documents: 48h
	\item Approval of documents: 48h
	\item Answer to questions: 24h
	\item Sending of documents: 24t
\end{itemize}

\subsubsection{Testing}

Test driven development will be performed. Unit tests will be performed, using the unit testing framework Jasmine, to ensure that the source code works as planned. To evaluate the product from a user perspective usability testing will be performend. System testing will be perfomed to evaluate whether the system complies with its specified requirements.

\subsubsection{Revision Control}

To make it possible for the group members to simultanously work with the same files without causing inconsitency, and to easily be able distribute own work to other group members Git will be used as version control system. Repositories for both code and the report have been established. All group members will work on their own branch when writing code and the name of the branch should explain what part of the product is in development. The master repository should always contain working code. Commits with meaningful comments should be made frequently.

More information on Git can be found in section 3.6.1.