\section{Methodology}

    This section briefly describes the chosen methodology for this project and how 
    it has been adapted into the project. The choice of methodology is based on the
    section in 'Preliminary Studies' about methodology.

\subsection*{Why is scrum suitable for this project?}

    In the start of this project, we didn't get handed a full requirement specification 
    or a complete game concept. The customer knew that they wanted a game, but handed a 
    lot of the responsibility to the team to carry out this. With this responsibility, 
    the team needed to choose a methodology that supported this challenge. The group 
    wanted to choose a methodology that would support the possibility of changes to the 
    requirement specification during the project. 

\subsection*{Scrum in our Project}

    This project will not follow the methodology to the letter. The reason for this is 
    that the team do not have the opportunity to work together as much as would be 
    preferred, with all members present. This affects for example the standup routine. 
    Standups will not be held every day, the way one would do in the scrum, but when the 
    team sit together every member gives a status report of what they are working on. 
    Even though standups are not held, it is still important to give the other team 
    members a status update.

    Allocation of roles has been limited to the Scrum master and the Scrum team where 
    each person in the scrum team has an area of responsibility. Sprint planning is run 
    in advance of each sprint and a sprint delivery is held after each sprint. The 
    objective of the sprint delivery is that the customer is able to see the results 
    and the progress of the project. During these deliveries the customer has the 
    opportunity to present feedback and changes if they want to.

    After each sprint a "sprint retrospective" is held to always strive for better working 
    process. This allows the group members to give feedback to each other.

