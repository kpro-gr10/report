\subsection{Methodology}

In this section we will briefly describe the methodology we have chosen 
for this project. We will also described how the methodology works and 
how we have adapted the methodology in to the project.
The prestudy on why we have chosen Scrum is described 
under the 'preliminary studies'. 



{\bf What is a methodology? } A project methodology is a approach a project can take to manage all the project activites. There are several types of approaches to take like iterative, incremental, sequential, etc. Regardless of the approach, every project has activites to complete, but it can be scheduled differently. 

{\bf What is a agile methodology? } In a 'normal' project cycle we have activites like analysis, planning, implementation, testing and delivery. How a project schedule these activites depends on the approach. 
The agile approache focus on helping the team to respond to unpredictability through incremental, iterative work cadences, known as sprints/pahses. Agile methodologies are an alternative to waterfall, or traditional sequential development.
Agile development has a known manifest called 'The agile manifesto' that works like this:

\begin{itemize}
	\item Individuals and interactions over processes and tools
    \item Working software over comprehensive documentation
    \item Customer collaboration over contract negotiation
    \item Responding to change over following a plan 
\end{itemize}

The agile manifesto is based on twelve principle that is inportant to understand before
adapting a agile methodology to the project:

\begin{enumerate}
    \item Customer satisfaction by rapid delivery of useful software
    \item Welcome changing requirements, even late in development
    \item Working software is delivered frequently (weeks rather than months)
    \item Working software is the principal measure of progress
    \item Sustainable development, able to maintain a constant pace
    \item Close, daily cooperation between business people and developers
    \item Face-to-face conversation is the best form of communication (co-location)
    \item Projects are built around motivated individuals, who should be trusted
    \item Continuous attention to technical excellence and good design
    \item Simplicity—the art of maximizing the amount of work not done—is essential
    \item Self-organizing teams
    \item Regular adaptation to changing circumstances
\end{enumerate}


{\bf What is scrum? }


{\bf How do we do scrum in the project? }


%http://en.wikipedia.org/wiki/Agile_software_development