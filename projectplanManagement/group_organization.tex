\clearpage
\section{Group Organization}

\begin{wrapfigure}{r}{31mm}
  \begin{center}
  \includegraphics[scale=0.72]{pictures/team.png}
  \end{center}
\end{wrapfigure}

The group consists of only 4 people, and this will affect how the group is organized. When roles and responsibilities are assigned, it will require that each group member has more than one role. In scrum there are mainly 3 roles: product owner, scrum master and the scrum team. Because the group is small there will be an overall flat structure. That means that
there will be traditionally project roles that will not be assigned to a specific person, but to the group as a whole. It will be the project managers responsability to ensure that activities within an unassiged role are covered during \\ the project.

\subsection{Roles}

{\bf The scrum master (project manager)} is responsible for the progress of the project as well as the organizational tasks like planning and workload.To keep the project moving in the right direction, the project manager has to make plans for activities like sprints, meetings and deliveries. He/she must maintain contact with the customer and the supervisor.
Before every sprint, meeting and delivery, the project manager has the responsibility to book a room, contact all
participants as well as creating the agenda for the meeting. \\

\noindent
{\bf The Scrum Team}
\begin{itemize}

  \item {\bf Test manager:} has the responsibility to ensure the quality of the product. To keep the quality of the product, the test manager has to ensure that each developer create unittests for the produced code, ensure that all the planned test activities is performed and that the results from all test activities is reported.

  \item {\bf Document owner:} has the responsibility to distribute tasks to all the group members so that the deadlines can be met. The document owner also has the responsibility to keep track of what sections in the report to deliver in each sprint and the quality of what is delivered.

  \item {\bf Technical leader:} has the role to ensure that the right technology is chosen, must have the overview of the version control system and the responsibility to train the group members in the technology needed.

  \item {\bf Development leader:} has the responsibility to ensure that the goals are met for each sprint and that each developer is doing their tasks. The goals are described in each sprint as "Features to Implement". 

  \item {\bf Developer:} each developer has the responsibility to produce quality code and write unittest for what he or she produces. Every developer also has the responsibility to make sure that the produced code that is pushed to the repository is running and working according to the requirement specification.

  \item{\bf Architect:} the software architect's responsibility as a computer programmer is to make a high-level design choices and dictates technical standards for the product developed. \cite{architect}

  \end{itemize} 

\subsection{Allocation of roles}
\begin{wrapfigure}{l}{39mm}
  \begin{center}
  \includegraphics[scale=0.7]{pictures/Work.png}
  \end{center}
\end{wrapfigure}

In this project, every member has a role with the responsibilities described in the section above ("Roles").
The roles were selected based on the needs in the project. Because the group is small, too many roles were not desirable.

There will always be small roles to fullfill, not all roles will be allocated in detail, but there will be a responsible person for the critical parts in the project. If roles needs to be allocated during the project, it will be the project managers job to do so.

The roles were allocated during a group meeting:
\begin{itemize}
  \item {\bf The Scrum Master (Project manager):} Marte Løge
  \item {\bf Test manager:} Solveig Hellan
  \item {\bf Document owner:} Solveig Hellan
  \item {\bf Technical leader:} Anders Wold Eldhuset
  \item {\bf Development leader:} Martin Stølen
  \item {\bf Architect:} Martin Stølen
  \item {\bf Developer:} Marte Løge, Solveig Hellan, Anders Wold Eldhuset and Martin Stølen
\end{itemize}





