%TODO: add a description of M, H and L in the risk analysis

\pagestyle{fancy}
\clearpage
\section{Risk Management}

This section will focus on the risks that have an attached probability and consequence to the project. If a risk occurs, it is important to have a plan for how to manage the risk and a description of how to act in a proactive and preventive way. The risk table was developed in the start of the project and may change throughout the project phases.

Below is two tables that describes the attributes 'consequence' and 'probability' that are attached to a risk:

\definecolor{gray}{gray}{0.6}

%Consequnce
\begin{table}[H]
\begin{tabular}{| p{3cm} | p{8cm} |}
  \hline
  \rowcolor{gray}
  {\bf Title} & {\bf Description} \\ \hline
    High (H) & A major problem, significant revision in plan or product. 
    A serious delay in deliverables.\\ \hline
    Medium (M) & Project could be delayed, a lot of work to meet deadlines but
    manageable.\\ \hline
    Low (L) & Minor impact on the project, less significant milestones not met.\\ \hline
\end{tabular}
\caption{Description of Consequence levels}
\end{table}

%Probability
\begin{table}[H]
\begin{tabular}{| p{3cm} | p{8cm} |}
  \hline
  \rowcolor{gray}
    {\bf Title} & {\bf Description} \\ \hline
    High (H) & Current circumstances show that the the risk is very 
    likely to occur.\\ \hline
    Medium (M) & Likely to occur.\\ \hline
    Low (L) & Unlikely to occur, and the circumstances likely to trigger 
    the risk are also unlikely to occur.\\ \hline
\end{tabular}
\caption{Description of possibility levels}
\end{table}

\pagestyle{empty}
\begin{landscape}
	\addtolength{\oddsidemargin}{-.8in}
	\addtolength{\topmargin}{0.5in}
  \begin{table}
	\begin{tabular}{| c | p{1.5cm} | p{4cm} | p{4cm} | c | p{4cm} | p{4cm} | c |}
    \hline
    \rowcolor{gray}
   	{\bf Nr} & {\bf Activity} & {\bf Risk factor} & {\bf Consequences} & {\bf Prob.} & {\bf Proactive strategy} & {\bf Reactive actions} & {\bf Responsible} \\ \hline

   	1 & All & Incomplete requirement specification from customer & H: Will result in a late start-up and more work later & M & REDUCE: Contact the customer early in the process. & Complete it ourselves and get approval from customer & Marte \\ \hline

   	2 & All & Samfundet and other voluntary work & M: Decrease in quality of project & H & EDUCE:Make sure we have clearly defined responsibilities & Clarify responsibilities and expectations through conversation & Marte \\ \hline

   	3 & Project delivery & Not reaching the expected quality of the customers expectations & H: Unhappy customer & M & Maintain contact with the customer throughout the project & Try to modify the customer’s expectations, or alter our end product if time allows it & Martin \\ \hline

   	4 & All & Illness & M: More work on others & H & ACCEPT & Postpone or distribute tasks to remaining group members dependent on importance of tasks & Anders \\ \hline

   	5 & Develop- ment and project delivery & Customer is not reachable. & M: May slow down or even halt the work process & L &
   	AVOID: Make clear appointments and keep the customer well informed & Keep trying to reach the customer & Solveig \\ \hline

   	6 & Develop- ment and testing & Other school work & M: Similar to point 2, a decrease in quality due to lowered effort& H & REDUCE: Again, similar to point 2: Make sure we have clear expectations. & 
	Clarify expectations through conversation. & Solveig \\ \hline

   	7 & Develop- ment & Continuous change to requirementsby customer & M: A slower and more stressful work process & M & AVOID: Sign a specification before work begins & Adapt if possible, otherwise refer to the signed specification & Martin \\ \hline

   	8 & Develop- ment and testing & Unforeseen problems with the development platform & 
	M: Slower work process and decreased quality of the end product & M & REDUCE: Do proper research and frequent testing (manual or automated) & Start again from a point where the work looked promising &Anders \\ \hline

   	9 & Develop- ment and project delivery & Unrealistic expectations & H: A difficult development process and an unhappy customer & L & AVOID: Create a clear specification which is approved and signed by both parties & Attempt to adjust the customer’s expectations through conversation & Marte \\
   	\hline
    \end{tabular}
    \end{table}
\end{landscape}

\pagestyle{fancy}
