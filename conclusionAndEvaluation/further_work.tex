\section{Further work}

This section describes various features that could be implemented in order to have a 
more user friendly and enjoyable application, but that have not been implemented in 
this project due to time limits and/or low priority.

\textit{Placing of buildings, power plants and power lines}

In the final version of the game a power plant may be built quite close beside a building 
and a building may appear quite close to an already built power plant. When building a power 
line from the power plant to the building, it can turn out quite short and it may be difficult 
to click on. A solution would be to have a constraint on where buildings may appear relative 
to power plants, as well as where it is possible to build power plants. Another solutions 
could be to somehow make the power line bigger and more clickable despite the short distance. 

Another problem in terms of were game elements are placed on the game map is the fact that 
there are no constraints on where power lines may be built as long as they are drawn between 
two buildings, either power plants or general buildings. There are either no constraints in 
where other buildings are placed in relation to a power line. This means that buildings may 
appear on top of a power line, which consequently can make the power line difficult to click 
and also make the screen look cluttered. Similarly it is possible to build a power plant on 
top of a power line or build a power line under a building or power plant. Power lines always 
appear under buildings. A solution could be to make a constraints on where power plants 
may be built and where buildings may appear in relations to power lines, or make constraints 
to where power lines may be built. Making the power line appear over power plants will in return 
make the power plant difficult to click. An easier solution could be to have a 
"Power line view mode" as stated in requirement 6.14, where buildings are not rendered and 
cannot be interacted with, but which makes power lines more visible and easier to interact with.
This requirement was not implemented due to time limitations and because it was added to the 
requirement specification at a late point.

\textit{Zooming}

The feedback from the usability test showed that several of the users wished it had been 
possible to build also in zoomed out mode. This feature has not been implemented due to 
time limitations. A possibly better way to handle zooming as opposed to double tapping, 
could be gradually zooming by using two fingers. These two features implemented together 
could lead to a more user friendly game.

\textit{Visualization}

Most of the users in the usability test had a problem understanding that a damaged power 
line was in fact damaged. That a power line is damaged could be indicated better by making 
it appear to be cut for instance.

The users also had difficult understanding that a power plant could be upgraded. This could 
be visualized better by having an arrow above the power plant, which then could be clicked in 
order to upgrade.

Visual notification?

\textit{Other}

Functional requirements 6.6 stated that a power preservation tip should appear when the user 
reached a new level. This had a low priority, because the customer stated at the first meeting 
that this was not important to the game, and has not been implemented. It is however room for a 
power preservation tip on the screen that appears after the user has completed a level, and it 
should be easy to implement this.

The customer stated in the sprint 3 delivery meeting that it would be nice to be able to build 
several power lines in a row without having to press the power line icon each time. This got 
somewhat easier after making the building icons visible at all time, but could be supported even 
more by adding a persistent building mode for power lines. That is, when a user presses the power 
line building icon and has built the first power line, the building mode is not ended, and the user 
can continue building without pressing the build power line icon, until he or she exits the building 
mode by pressing an exit button.

\textit{Support/Compatibility}

Music that is supported on several versions of android and on ios(?)

Better support on different android phones.