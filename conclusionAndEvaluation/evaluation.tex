\section{Evaluation}

In this section we will discuss how the group dynamics have worked over the
course of the project and the interactions with the customer and the course
staff. We will discuss the course and what the group thinks about working on the
project.

\subsection{Internal Process and Results}

	%(How have you worked together as a team? What have you done 
	%well? What have you not done so well? What would you have done differently? Conflicts that 
	%arose and how these were handled? Did you reach the project goals? What did you learn?)

	In the start of the project we had some problems with the dynamics in the group. 
	All the team members are very different in terms of how we like to work, how we communicate, 
	our personalities as well as our experience with this kind of project. Therefore communication 
	did not go so well and people were unsure of what were expected of them. We managed to solve 
	this problem by discussing the problem and discovered that one of the main problems was that 
	the role allocation and the responsibility had not been present. After we made clear what the 
	responsibility to each role was, many of our problems were solved and we managed to work better 
	as a team.

	In most of the sprints we did not manage to reach the goal of working 20-25 hours a week in 
	average per person. However most of the group members did managed to reach this goal individually 
	for most of the weeks. A lot of voluntary work was the reason that some members struggled to reach 
	this goal. Nevertheless, apart from a few minor ones, we completed the requirements in the specification.

	Several of the risks outlined in section 1.8 Risk Management did occur. We received an incomplete 
	requirement specification from customer and handled this by making a complete specification that 
	was approved by the customer, and all in all this did not affect the start-up of the project too much. 
	Some of the members took part in a lot of voluntary work at, among other places, Samfundet. This was handled 
	by giving specific tasks that could be worked on outside group meeting times, but did still lead to fewer 
	working hours. Other school work also affected the work hours, but we tried to work effectively those hours 
	that were spent on the project.

	Unit tests were unfortunately not written. Although we do understand the importance of them, 
	we felt that we needed one more person on the group to make it possible, because we could not 
	find the time and resources for it with the four of us. On the other hand, the functional
	testing helped discovering a lot of bugs and errors on its own.  
	Device testing or compatibility testing was not performed either. We thought we would have more 
	time at the finishing stage of the project than we ended up having.
	We were however very happy with the execution and feedback from the usability test, which proved 
	to be very useful to us.


\subsection{The Customer and Project Task}

	%(How was the communication with the customer? How did you experience the project assignment?)

	The customer is based in Mosjøen, while the team was in Trondheim. This lead us to having the 
	meetings over Skype and it took us some time to find a good way to make this work. In the beginning 
	we had some trouble finding a good way to show the customer the app and let them interact with it, 
	so that they could give us proper feedback. We provided them with a QR-code that let them install 
	the app on their own phones. This did to a certain extent make up for the fact that we could not 
	be in the same room for a meeting.

	We thought the project task sounded very interesting as it would allow us to learn to make a mobile app, 
	something which none of the team members had done before, but which we wanted to be able to do. Also, 
	the team members enjoy playing games themselves, so making a game proved to be a fun kind of task. 
	We did however feel that the task was a bit vague in the beginning and struggled with finding a good 
	concept for a game concerning the power industry, as neither one of us had any extensive knowledge 
	in this field. However when we had agreed on a concept with the customer and the implementation could 
	start we were able to progress without much uncertainty, and no major changes need to be made to the
	game concept after we had begun the implementation.


\subsection{The Advisor}

	%(How was communication with the advisors? Was the supervision good enough? 
	%How could the course be improved to next year?)

	The advisor was engaged in our project and let us know early on if he thought we were facing difficulties or would be and how to best handle these. He gave us a lot of valuable feedback and quickly responded to questions we had outside of the meetings.

\subsection{Suggestions for Improvements}

	We do not know how much communication there was between the course staff and the customers 
	before the start of the course, but we had the impression that several of the customers that 
	took part in the course for the first time did not know what to expect from the project and 
	what was expected of them. Perhaps providing new customers with more information on what to expect 
	and more opportunities to ask questions before the start of the course would make it easier for 
	them to take part in it.

	The composition of the teams could be based on what fields the student have specialized themselves 
	in and the task given. We felt we had a good balance in our group, but other groups may have had a 
	different experience.

%Annet:

%\subsection{Project}


%\subsubsection*{Communication with the customer}

	



%challenges or difficulties?

%communication supervisor?
%group dynamics and role allocations?
%risks?
%goals met/not met?

%\subsection{(Development) Process}

%scrum?
%hours?
%testing?


%\subsection{Implementation}

%backbone?
%phonegap?
%javascript?

%\subsection{Final product}
