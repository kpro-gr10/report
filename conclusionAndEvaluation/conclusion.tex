\section{Conclusion}

All in all, we (the group) feel that this project was a valuable learning
experience. We've experienced at least some of the challenges software
developers working with customers typically meet, we've gained experience
in practical team work and we've also gained new technical insights. These
experiences were amongst our personal goals for the project, alongside making
a product which would please the customer. It is our impression that we have
achieved this latter goal as well.

In terms of real-world challenges, we've dealt with incomplete project
requirements, somewhat challenging group dynamics and unknown technologies.
The problem of incomplete requirements was solved by a combination of creative
thinking and good relations with our customer, which ultimately lead to a
finalized concept which was acceptable to both the customer and the group.
As for group dynamics, we met with the situation that different team members
have different work routines and personalities, a very real possibility when
being placed on a project by course staff or professional superiors. This
caused some slight growing pains, but as the team members got to know each
other and adapted accordingly, it ceased to be a problem. As for learning new
technologies, this is to be expected in a real project and, while it may be
challenging at times, it did not represent a problem in our case.

If we were to emphasize one insight we've gained from this project, it is that
most of our serious challenges were overcome through communication. Whether it
be within the group, with our supervisor or with the customer, problems and
uncertainties are often more easily identified through discussion than through
individual thought, and a problem cannot be solved before it is identified.
