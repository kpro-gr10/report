\section{Project Description}

This project is a part of the NTNU course TDT4290 Customer Driven Project, which is mandatory for all 4th year students at the Department of Computer and Information Science. The purpose of the course is to give the students experience in carrying out an IS/IT project, making it as close to a real life working situation as possible. This involves, among other factors, planning and managing the project, developing, maintaining customer relations and establishing good group dynamics, and evaluating the project.

The purpose of this specific project is to make a casual game for iOS and Android. The game should focus on controlling power production and distribution of power to the customers. The aim is to arouse interest in what the power business do as well as promoting the customer Helgelandskraft.

The description of the problem as stated by Helgelandskraft, and found in the course compendium is given below: 
		\paragraph{Power Control Game}
		{\it The idea is to make a casual game for mobile devices focused around controlling 
		power production from hydro plants trough a power grid to large industry customers and 
		regular consumers, or some other casual game involving power grids, hydro plants 
		and/or other themes around hydro power. 
		 
		The application should also provide the user with power preservation tips, like a 
		tip on loading screens or the splash screen when opening the game. Tips like: 
		 
		“19-21 degrees celsius is a good inside temperature. For each degree you lower the temperature 
		you’ll save 5 procent of the energy used for heating up your living space. 
		You can lower the temperature even more in rooms you normally don’t use.” 
		 
		The application should be developed for iOS and Android. We’ll provide the 
		students with the required equipment needed for iOS-development if they don’t 
		have it, and an android phone if they need it. 
		 
		The application should be designed for the tips and the language to easily be changed.}

\clearpage