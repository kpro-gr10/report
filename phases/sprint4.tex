\section{Sprint 4}

\subsection{Sprint planning}

	This is the last sprint where new features will be implemented. We plan to implement the last 
	requirements which are all rated medium and low. Several necessary improvements and some new 
	features were discovered during the usability test held in sprint 3, and these will be implemented 
	in this sprint. Some time will also be spent fixing bugs.

\subsection{Duration and Workload}

\clearpage
\subsection{Sprint backlog}
	\begin{table}
	\begin{tabular}{| p{1.2cm} | p{8cm} | p{3cm} |}
		\hline
		\rowcolor{gray}
		ID & Description & Estimate \\ \hline
		FR1.2 &  The user should be able to exit the game at any time, and the game state should be saved and loaded when the user returns to the game. & \\ \hline

		FR1.4 & The user should get a high score based on the time spent finishing the level. & \\ \hline

		FR1.5 & The user should be able to improve the high score on any level by beating previous high scores. & \\ \hline

		FR1.6 & The user should be able to pause the game at any time. & \\ \hline

		FR1.7 & The game should play background music. & \\ \hline

		FR1.8 & The user should be able to turn on/off the background music. & \\ \hline

		FR1.9 & There should be played a sound effect when the player collects money. & \\ \hline

		FR1.10 & There should be played a sound effect when the player upgrades the powerplant. & \\ \hline

		FR1.11 & There should be played a sound effect when the player builds a power cable & \\ \hline

		FR1.13 & There should be a sound notification for when new obstacles appear. & \\ \hline

		FR1.14 & There should be played a sound effect when the player builds a power plant. & \\ \hline

		FR2.7 & The player should only be allowed to build a level-specific number of power plants. & Not implemented. \\ \hline

		FR2.9 & The houses should not appear on top of each other. & \\ \hline
		
		FR6.6 & A power preservation tip should appear when the user reaches a new level. & Not implemented. \\ \hline

		FR6.14 & The user should be able to click a button, and go into "powerline view mode", where the buildings aren't rendered and cannot be interacted with. This is to make it easier to see power lines behind buildings etc. & Not implemented. \\ \hline
	\end{tabular}
	\caption{Sprint backlog sprint 4}
	\end{table}

\subsection{Implementation}

	\subsubsection*{Pause game}

	\subsubsection*{Highscore}
	
\begin{figure}[H]
	\centering
	\subfigure{
		\includegraphics[scale=0.18]{pictures/sprint4-screen/splashscreen}
	}
	\subfigure{
		\includegraphics[scale=0.18]{pictures/sprint4-screen/mainmenu}
	}
	\caption{Splash screen when game starts and the game's main menu}
\end{figure}

\begin{figure}[H]
	\centering
	\subfigure{
		\includegraphics[scale=0.18]{pictures/sprint4-screen/highscore}
	}
	\subfigure{
		\includegraphics[scale=0.18]{pictures/sprint4-screen/instructions}
	}
	\caption{Highscore and instruction screens}
\end{figure}

\begin{figure}[H]
	\centering
	\subfigure{
		\includegraphics[scale=0.18]{pictures/sprint4-screen/game_start}
	}
	\subfigure{
		\includegraphics[scale=0.18]{pictures/sprint4-screen/mapoverview}
	}
	\caption{Message when starting a new level and an overview over the map during gameplay}
\end{figure}

\begin{figure}[H]
	\centering
	\subfigure{
		\includegraphics[scale=0.18]{pictures/sprint4-screen/buildpowerplant}
	}
	\subfigure{
		\includegraphics[scale=0.18]{pictures/sprint4-screen/upgradepowerplant}
	}
	\caption{Build and upgrade powerplants}
\end{figure}

\begin{figure}[H]
	\centering
	\subfigure{
		\includegraphics[scale=0.18]{pictures/sprint4-screen/buildpowercable}
	}
	\subfigure{
		\includegraphics[scale=0.18]{pictures/sprint4-screen/buildpowercable2}
	}
	\caption{Building a powercable}
\end{figure}

\begin{figure}[H]
	\centering
	\subfigure{
		\includegraphics[scale=0.18]{pictures/sprint4-screen/gameover}
	}
	\subfigure{
		\includegraphics[scale=0.18]{pictures/sprint4-screen/victory}
	}
	\caption{Gameover and victory screens}
\end{figure}
\subsection{Redesign and bug fixing}
	
	\subsubsection*{Changes to hud}
		**Made the hud static**
		After receiving feedback from the customer and through observations during the usability test in sprint 3, it was decided that the hud was going to be static, i.e. it was not going to be necessary to move it up in order to build.The two icons for building power plants and power lines were made slightly smaller to not make the screen look cluttered.
	\subsubsection*{Scaling GUI elements on Ios}
		**Make the elements fit on ios**
	\subsubsection*{Improvements on the appearance of buildings}
		**Changed map**
		A grid was added to the map...
	\subsubsection*{Input handling on Ios}
		**Fungerer nå på android og Ios**

	\subsubsection*{Changed Game logo}
		We did not have the font size of the first logo, so we changed it to a new 
		and better looking one. 

		\begin{figure}
			\centering
			\includegraphics[scale=0.4]{pictures/logo2.png}
			\caption{Old logo}
		\end{figure}

		\begin{figure}
			\centering
			\includegraphics[scale=0.4]{pictures/newLogo.png}
			\caption{New logo}
		\end{figure}


\subsection{Testing}

	The following test cases were executed at the end of this sprint:

	\definecolor{lightgray}{gray}{0.9}

	\begin{tabular}{| p{3cm} | p{7cm} | p{2cm} |}
		\hline
		\rowcolor{lightgray}
		{\bf Test Case} & {\bf Result} & {\bf Pass/Fail} \\ \hline

		FT-25 High score & Works as expected. & Pass. \\ \hline

		FT-26 Pause Game & If one leaves the game while paused the alert follows to the main menu. In order 
		to get anything done in the main menu one need to press 'Resume', which makes the game continue while 
		the player is in the main menu. It can be intuitive for the player to pause the game before going to 
		the main menu, but in the case of our game this works better by just leaving the game. Otherwise the 
		pause function works. & Partially pass. \\ \hline

		FT-27 Sound Effects & Sound does not work on some devices. & Partially pass. \\ \hline

		FT-28 Exit Game & If the game is left when a level is completed, but before the next level has been 
		entered, the game returns to main menu and the player needs to start at level 1 again. When exiting 
		the game while in playing mode, it works. & Partially pass. \\ \hline



		%FT-32 Power Preservation Tip &  & \\ \hline 
	\end{tabular}

	The following test cases did not pass previous tests and were retested this sprint:

	\begin{tabular}{| p{3cm} | p{7cm} | p{2cm} |}
		\hline
		\rowcolor{lightgray}
		{\bf Test Case} & {\bf Result} & {\bf Pass/Fail} \\ \hline
		FT-02 Appearance of buildings & Buildings do not appear on top of each other anymore. & Pass. \\ \hline
		FT-05 Build Power Plants & Power Plant are not possible to place on top of other buildings. & Pass. \\ \hline
		FT-06 Build Power Lines & There is still no constraints on where power lines can be built as long as it is between two buildings. Otherwise they work. & Partially pass. \\ \hline
		FT-20 Damaged Power Lines & Power lines are now possible to fix. & Pass. \\ \hline
	\end{tabular}

	The test cases that are marked with 'partially pass', works mostly as expected, but has some bugs 
	that we will not be able to fix before delivery.

	Other bugs that was were discovered during testing was that if a building or a power plant is placed 
	too far on the right side on the game map, their bar is not visible to the player. Also after having 
	changed the styling of the dialog box for building power lines, it was no longer possible to decline 
	building a power line when asked.

\subsection{Changes to the requirements}

	During testing early in the sprint it was discovered that despite the improvements on the appearance of buildings, i.e.the grid, new buildings could still appear on top of power lines, something which complicates removing and fixing these power lines, as well as making a clutter when building power lines to the buildings. An optimal solution would be to make another constraint on were buildings can appear regarding the power lines. As this was the last sprint and fixing this could be potentially difficult and time consuming, the group decided on another solution to this problem as seen  in requirement FR6.14.

	The fact that buildings should not appear on top of each other has finally been added as a requirement, and an additional sound effect requirement has been added.

	Requirement FR1.3 has been removed. The initial plan was that the player could play any level as long as he or she had completed the previous one, and chose which level to play from a menu of all levels. This would be demanding to implement and in addition the levels do not change character to a great extent, only the parameters are changed, from level to level. Therefore it was decided that we will operate with 'tetris-levels', i.e. the player always starts at level 1 and can enter higher levels by winning all previous levels without loosing. 

	{\bf Changes to version 4 of the requirement specification:} \\
	\begin{tabular}{| p{1.5cm} | p{12cm} |}
		\hline
		\rowcolor{lightgray}
		{\bf FR} & {\bf Change} \\ \hline
		FR1.3 & {\bf \color{red}[REMOVED]}The user should be able to restart any level at any given point of time after start playing. \\ \hline
		FR1.14 & {\bf \color{green}[NEW]} There should be played a sound effect when the player builds a power plant. \\ \hline
		FR2.9 & {\bf \color{green}[NEW]} The houses should not appear on top of each other. \\ \hline
		FR6.14 & {\bf \color{green}[NEW]} The user should be able to click a button, and go into "powerline view mode", where the buildings are not rendered and cannot be interacted with. This is to make it easier to see power lines behind buildings etc. \\ \hline
	\end{tabular}

\subsection{Group dynamics}

\subsection{Customer feedback}

\subsection{Sprint retrospective}
