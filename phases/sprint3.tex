\section{Sprint 3}

\subsection{Sprint planning}

\subsection{Expected sprint results}
	In this sprint, the group expect to be finished with the most of the requirements 
	with the critical and high priority. Sprint 4 is the last sprint, so it is important
	to only work with bugfixing and implementation of requirements with low priority.


\subsection{Duration and worklaod}

\clearpage
\subsection{Sprint backlog}

	\begin{tabular}{| p{1.2cm} | p{8cm} | p{3cm} |}
		\hline
		\rowcolor{gray}
		ID & Description & Estimate \\ \hline

		FR2.5 & The amount of power available to the user should be limited; 
		the power supply of a power plant should be upgradeable. & \\ \hline

		FR2.6 & The user should be able to remove power lines from the 
		& \\ \hline

		FR2.7 & The player should only be allowed to build a level-specific 
		number of power plants. & \\ \hline

		FR3.1 & Arbitrary power lines may be damaged throughout the game & \\ \hline

		FR3.2 & The user should be able to fix unstable power lines before it is 
		broken; this should cost some amount of money. & \\ \hline

		FR3.3 & There should be several types of buildings on the map, with different 
		power requirements & \\ \hline

		FR3.4 & Different types of building should reward different amounts of 
		money & \\ \hline

		FR4.1 & The user should be able to continue to the next level when the goal is 
		reached & \\ \hline

		FR4.3 & As the user reaches higher levels new buildings appear more 
		rapidly & \\ \hline

		FR4.4 & As the user reaches higher levels unstable power lines will appear 
		more rapidly & \\ \hline

		FR4.5 & As the user reaches higher levels the map size may increase & \\ \hline

		FR4.6 & The user should be able to win the game by reaching the goal in 
		the current level. The goal is level specific. & \\ \hline

		FR5.2 & When connecting buildings through power cables, there should be a 
		cost which is proportional to the length of the cable. & \\ \hline

		FR6.11 & The user should be able to see which houses is selected when 
		building power cables & \\ \hline

		FR6.12 & The cables should change color if it is connected to a power 
		station. & \\ \hline

		FR6.13 & The user should be able to see how much power the powerplant have 
		left. This bar should decrease if a building is connected to the powerplant 
		and should be increase if a building is removed. The colors should be yellow 
		with white background. & \\ \hline

	\end{tabular}

\subsection{Implementation}

	Write about algorithms and implementations done!
	
	\subsection*{Algorithms}

		\subsubsection*{Breath First Search}
			Breath First Search is a standard algorithm taught in both Discrete Mathematics and 
			Algorithm courses. Breath First Search (from now on refered to as BFS) traverses a graph 
			by exploring one level of the graph at a time. BFS is used to distribute power from the 
			powerplants to the buildings on the map. The graph is defined by having all buildings 
			and powerplants be vertices and the powerlines are edges. The algorithm runs once for 
			each powerplant on the map every time there is a change to the graph.

		\subsubsection*{Point to Line Segment Distance}
			The Point to Line Segment Distance works by projecting the point \emph{p} onto the line 
			defined by going through the point \emph{v} and the point \emph{w}. Then the algorithm 
			examines the following three different locations for the projection on the line: 
			\begin{enumerate}
				\item Before \emph{v}.
				\item After \emph{w}.
				\item Between \emph{v} and \emph{w}.
			\end{enumerate}
			If the projection lies before \emph{v} on the line, then the distance from \emph{p} to 
			the line segment from \emph{v} to \emph{w} is simply the distance from \emph{p} to 
			\emph{v}. If the projection lies after \emph{w} on the line, then the distance from 
			\emph{p} to the line segment is the distance from \emph{p} to \emph{w}. In the last case 
			where the projection lies between \emph{v} and \emph{w}, the answer is the distance from 
			\emph{p} to the location of the projection. There is also the special case where the 
			length of the line segment from \emph{v} to \emph{w} is 0. In that case the distance 
			from \emph{p} to the line segment is just the distance from \emph{p} to either \emph{v} 
			or \emph{w}.

			\begin{figure}[H]
			\centering
			\subfigure{
				\includegraphics[scale=0.75]{pictures/PtLSplot}
			}
			\caption{Plot of Point to Line Segment Distance}
			\end{figure}

			Theory source: \url{http://math.stackexchange.com/a/322836} \newline
			Code source: \url{http://stackoverflow.com/a/1501725} \newline

\subsection{Testing}

	\definecolor{lightgray}{gray}{0.9}

	\begin{tabular}{| p{2cm} | p{7cm} | p{3cm} |}
		\hline
		\rowcolor{lightgray}
		{\bf Test Case} & {\bf Result} & {\bf Pass/Not pass} \\ \hline

	  	FT-06 Build Power Lines &  &  \\ \hline

	  	FT-11 Collect Money &  &  \\ \hline

	  	FT-14 Win Game &  &  \\ \hline
	  	
	  	FT-16 Limited Power &  &  \\ \hline
	  	
	  	FT-17 Remove Power Cables &  &  \\ \hline
	  	
	  	FT-18 Number of Power Plants &  &  \\ \hline
	  	
	  	FT-19 Different Buildings &  &  \\ \hline

	  	FT-20 Damaged Power Lines &  &  \\ \hline
	  	
	  	FT-21 Enter next level &  &  \\ \hline

	  	FT-22 Appearance of buildings in new levels &  &  \\ \hline

	  	FT-23 More unstable power lines in new levels &  &  \\ \hline

	  	FT-25 Selected houses &  &  \\ \hline

	  	FT-26 Color of power cable &  &  \\ \hline

	\end{tabular}

\subsection{Changes to the requirements}
	
	{\bf Changes on version 3 of the requirement specification:} \\
	\begin{tabular}{| p{1.5cm} | p{12cm} |}
		\hline
		\rowcolor{lightgray}
		{\bf FR} & {\bf Change} \\ \hline
		FR 4.6 & {\bf \color{green}[NEW]} The user should be able to win the game by reaching the goal in the current level. The goal is level specific. \\ \hline
		FR6.11 & {\bf \color{green}[NEW]} The user should be able to see which houses is selected when building power cables. \\ \hline
		FR6.12 & {\bf \color{green}[NEW]} The cables should change color if it is connected to a power 
		station. \\ \hline
		FR6.13 & {\bf \color{green}[NEW]} The user should be able to see how much power the powerplant have left. 
		This bar should decrease if a building is connected to the powerplant and should be 
		increase if a building is removed. The colors should be yellow whit white 
		background. \\ \hline
	\end{tabular}


\subsection{Group dynamics}

\subsection{Customer feedback}

\subsection{Sprint retrospective}
	\subsubsection*{Start doing: } 
	\subsubsection*{Stop doing: }
	\subsubsection*{Continue doing: }