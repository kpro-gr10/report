\subsection{About}
Game concept is a detailed description of the game, including information such as high concept, genre,
gameplay description, features, setting, story, target audience, hardware platforms, estimated schedule,
marketing analysis, team requirements and risk analysis. Source: 
\url{en.wikipedia.org/wiki/Game_development#Development_process}
Several of these points like the risk analysis and hardware platforms have been brought up in other 
sections of the report and will not be discussed here. This section will mainly be about the gameplay. 
The game concept is the basis for the game's functional requirement. It is easier to explain the game 
as well as decide on the functional equirements for the game on the basis of the game concept. 

\subsection{High Concept}
The name for the game we settled on is Power Supply. It is in the construction and management 
simulation genre and is inspired by games like SimCity, only in a more limited scope. You are 
tasked with managing the delivery of electricity to cities and factories in an area, while the 
ultimate goal of the game is making money as the tycoon of the local power company, without going 
bankrupt.

\subsection{Gameplay Description}

\subsubsection{Player}
The role of the player is to build powerplants to supply the buildings on the game map with power. 
To do this the player has to connect the powerplants to the buildings with powerlines. Either directly
or indirectly. Building powerplants and powerlines costs money. To earn money the player has to supply
power to the buildings on the map. If the inhabitants of a building does not get power then they will
move out and abandon the building. If too many buildings become abandoned then the game is over.
Abandoned buildings disappear from the game.

\subsubsection{Powerplants}
Powerplants can only supply a fixed amount of power to the connected buildings. If a powerplant is 
unable to supply power to all the buildings on the map, the player has three options. The first option 
is to build more powerplants. The second powerplant can then be connected to the buildings that the
player wasn't previously able to supply with power. The second option is to upgrade the powerplant,
so that it can deliver more power. This is supposed to be more cost effective in the short term, however
it can become more expensive in the long run because of other important factors which will be brought
up later. The final option is to demolish an already existing powerline and allow the powerplant to
supply another building with power. This could be risky, as buildings that are not supplied with power
will eventually become abandoned.

\subsubsection{Powerlines}
Powerlines is the way the player connects their powerplants to the buildings in need of power. This 
can be done directly or indirectly. A direct connection is a connection between a powerplant and a 
building in need of power. While an indirect connection is to connect a building with another building.
Power from the powerplant then travels through the directly connected powerlines and continues down 
the indirectly connected powerlines, supplying as many buildings as possible with power. Direct 
connections are important, since buildings that are directly connected will receive power first.
However you also have to consider the length of the powerline, as longer powerlines are more expensive
to build than short powerlines. You can also tear down powerlines in order to supply other buildings
with power.

\subsubsection{Buildings}

\subsubsection{Goal}

